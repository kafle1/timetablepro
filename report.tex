\documentclass[12pt,a4paper]{report}

% Packages
\usepackage[T1]{fontenc}
\usepackage{geometry}
\usepackage{setspace}
\usepackage{graphicx}
\usepackage{caption}
\usepackage{fancyhdr}
\usepackage[backend=biber, style=ieee]{biblatex} 
\addbibresource{reference.bib}
\usepackage{tocloft}
\usepackage{titlesec}
\usepackage{parskip}
\usepackage{fancybox}
\usepackage{longtable} 
\usepackage{array} 
\usepackage{booktabs}  
\usepackage{multicol}
\usepackage{tikz}
\usepackage{pgfgantt}
\usepackage{pgfplots}
\usepackage[american voltages, american currents]{circuitikz}
\usetikzlibrary{shapes,arrows,positioning,calc,mindmap,trees,backgrounds}
\usetikzlibrary{er}
\usepackage{forest}
\usepackage{listings}
\usepackage{color}
\usepackage{float}

% Define colors
\definecolor{codegreen}{rgb}{0,0.6,0}
\definecolor{codegray}{rgb}{0.5,0.5,0.5}
\definecolor{codepurple}{rgb}{0.58,0,0.82}
\definecolor{backcolour}{rgb}{0.95,0.95,0.92}
\definecolor{primarycolor}{RGB}{70,130,180}
\definecolor{secondarycolor}{RGB}{255,165,0}
\definecolor{accentcolor}{RGB}{60,179,113}

% TikZ styles
\tikzstyle{block} = [rectangle, draw, fill=primarycolor!20, 
    text width=5em, text centered, rounded corners, minimum height=4em]
\tikzstyle{line} = [draw, -latex']
\tikzstyle{cloud} = [draw, ellipse, fill=primarycolor!20, 
    minimum height=2em]

% Code listing style
\lstdefinestyle{mystyle}{
    backgroundcolor=\color{backcolour},   
    commentstyle=\color{codegreen},
    keywordstyle=\color{magenta},
    numberstyle=\tiny\color{codegray},
    stringstyle=\color{codepurple},
    basicstyle=\ttfamily\footnotesize,
    breakatwhitespace=false,         
    breaklines=true,                 
    captionpos=b,                    
    keepspaces=true,                 
    numbers=left,                    
    numbersep=5pt,                  
    showspaces=false,                
    showstringspaces=false,
    showtabs=false,                  
    tabsize=2
}
\lstset{style=mystyle}

% Formatting Chapter Titles
\titleformat{\chapter}[block]
    {\normalfont\Large\bfseries\centering}
    {CHAPTER \thechapter:}
    {0.5em}
    {\uppercase}

\titleformat{name=\chapter,numberless}[block]
    {\normalfont\Large\bfseries\centering}
    {}
    {0pt}
    {\uppercase}

\titlespacing*{\chapter}{0pt}{-1cm}{*}

% Table of Contents Formatting
\renewcommand{\cftchappresnum}{CHAPTER }
\renewcommand{\cftchapaftersnum}{:}
\setlength{\cftchapnumwidth}{7.5em}

\renewcommand{\contentsname}{\MakeUppercase{Contents}} 
\renewcommand{\cftchapfont}{\uppercase\bfseries} 
\renewcommand{\cftchappagefont}{\bfseries} 
\renewcommand{\cftsecfont}{\uppercase} 
\renewcommand{\cftsubsecfont}{\uppercase} 
\renewcommand{\cftsubsubsecfont}{\uppercase} 
\renewcommand{\cftsecpagefont}{} 
\renewcommand{\cftsubsecpagefont}{} 
\renewcommand{\cftsubsubsecpagefont}{}

% Page Layout
\geometry{
    a4paper,
    top=1in,
    bottom=1in,
    left=1.25in,
    right=1in
}

% Paragraph Style
\setlength{\parskip}{1em}
\renewcommand{\baselinestretch}{1.5}

% Page Numbering
\fancypagestyle{plain}{
    \fancyhf{}
    \fancyfoot[C]{\thepage}
    \renewcommand{\headrulewidth}{0pt}
}

\pagestyle{plain}

\begin{document}

% Title Page
\begin{titlepage}
    \centering
    {\LARGE \textbf{Lincoln University College}\par}
    \vspace{0.2cm}
    {\Large \textbf{Faculty of Computer Science and Multimedia}\par}
    \vspace{0.5cm}
    {\LARGE \textbf{TimeTablePro: A Modern Timetable Management System}\par}
    \vspace{0.5cm}
    {\Large \textbf{A Project Report}\par}
    \vspace{0.3cm}
    {\large Submitted to\par}
    \vspace{0.3cm}
    {\Large Saishab Bhattarai\par}
    {\Large Lecturer\par}
    {\Large Software Engineering [BIT 244]\par}
    \vspace{0.3cm}
    {\large In partial fulfillment of the requirements for the Internal Evaluation \par}
    \vspace{0.3cm}
    {\large Submitted by\par}
    \vspace{0.2cm}
    {\Large \textbf{Niraj Kafle}\par}
    {\Large BIT 5th Semester\par}
    {\Large University ID: LC0003001674\par}
    \vspace{0.5cm}
    {\large \textbf{2025/03/05}\par}
\end{titlepage}
\clearpage

% Table of Contents
\tableofcontents
\clearpage

\chapter{Introduction}
\section{Background}
In today's educational institutions, efficient timetable management is crucial for smooth operations \cite{timetable_systems}. Traditional manual scheduling methods are time-consuming, error-prone, and often result in conflicts. TimeTablePro addresses these challenges by providing a modern, automated solution for timetable management.

The education sector has seen significant digital transformation in recent years, yet many institutions still struggle with efficient schedule management \cite{scheduling_algorithms}. This project aims to bridge this gap by leveraging modern web technologies and intelligent algorithms to create a robust timetable management system.

\section{Problem Statement}
Educational institutions face several challenges in timetable management:
\begin{itemize}
    \item Manual scheduling is time-consuming and error-prone
    \item Difficulty in handling schedule conflicts
    \item Lack of real-time updates and notifications
    \item Inefficient resource allocation
    \item Poor accessibility and sharing of schedule information
\end{itemize}

\section{Objectives}
The primary objectives of TimeTablePro are:
\begin{itemize}
    \item Develop an automated timetable management system
    \item Implement intelligent conflict detection and resolution
    \item Provide real-time schedule updates and notifications
    \item Enable efficient resource allocation and management
    \item Create an intuitive user interface for all stakeholders
    \item Ensure system scalability and reliability
\end{itemize}

\section{Scope and Limitation}
\subsection{Scope}
The system encompasses:
\begin{itemize}
    \item Web-based timetable management interface
    \item Automated schedule generation and conflict detection
    \item User role management (Admin, Teacher, Student)
    \item Room and resource allocation
    \item Real-time notifications and updates
    \item Schedule export and sharing capabilities
\end{itemize}

\subsection{Limitations}
Current limitations include:
\begin{itemize}
    \item Limited support for complex recurring schedules
    \item No mobile application (web-responsive only)
    \item Maximum capacity of 10,000 concurrent users
    \item Requires stable internet connection
\end{itemize}

\section{Report Organization}
This report is organized into the following chapters:

\begin{enumerate}
    \item \textbf{Introduction:} Provides project background, objectives, and scope
    \item \textbf{Literature Review:} Examines existing solutions and theoretical foundations
    \item \textbf{System Analysis and Design:} Details system architecture and design decisions
    \item \textbf{Testing:} Describes testing methodologies and results
    \item \textbf{Conclusion:} Summarizes findings and future recommendations
\end{enumerate}

\chapter{Literature Review}
\section{Background Study}
\subsection{Timetable Management Systems}
Timetable management systems have evolved from manual paper-based systems to sophisticated digital solutions \cite{timetable_systems}. Key concepts include:

\begin{itemize}
    \item \textbf{Scheduling Algorithms:} Various approaches including:
        \begin{itemize}
            \item Genetic Algorithms
            \item Constraint Programming
            \item Graph Coloring
        \end{itemize}
    \item \textbf{Resource Allocation:} Efficient distribution of:
        \begin{itemize}
            \item Classrooms and Labs
            \item Teaching Staff
            \item Equipment
        \end{itemize}
    \item \textbf{Conflict Resolution:} Methods for handling:
        \begin{itemize}
            \item Time Conflicts
            \item Resource Conflicts
            \item Teacher Availability \cite{resource_allocation}
        \end{itemize}
\end{itemize}

\subsection{Modern Web Technologies}
The project utilizes current web technologies \cite{web_development}:

\begin{itemize}
    \item \textbf{Frontend:} SvelteKit with TypeScript
    \item \textbf{Backend:} SvelteKit API routes with Appwrite
    \item \textbf{Database:} Appwrite Cloud
    \item \textbf{Authentication:} Appwrite Authentication
    \item \textbf{UI Framework:} Tailwind CSS
    \item \textbf{Calendar:} FullCalendar.js
\end{itemize}

\section{Literature Review}
\subsection{Existing Solutions Analysis}
Review of similar systems reveals common patterns and limitations:

\begin{itemize}
    \item \textbf{Traditional Systems:}
        \begin{itemize}
            \item Manual scheduling
            \item Limited automation
            \item Poor scalability
        \end{itemize}
    \item \textbf{Modern Solutions:}
        \begin{itemize}
            \item Automated scheduling
            \item Cloud-based storage
            \item Real-time updates
        \end{itemize}
\end{itemize}

\begin{figure}[H]
\centering
\begin{tikzpicture}[scale=0.8]
\begin{ganttchart}[
    vgrid,
    hgrid,
    bar/.style={fill=primarycolor},
    milestone/.style={fill=red},
    x unit=0.7cm,
    y unit chart=0.7cm
]{1}{12}
\gantttitle{Evolution of Timetable Systems}{12} \\
\gantttitlelist{1,...,12}{1} \\
\ganttbar{Manual Systems}{1}{3} \\
\ganttbar{Basic Digital}{3}{6} \\
\ganttbar{Web-Based}{5}{8} \\
\ganttbar{Cloud Solutions}{7}{10} \\
\ganttbar{AI-Powered}{9}{12}
\end{ganttchart}
\end{tikzpicture}
\caption{Evolution of Timetable Management Systems}
\label{fig:evolution}
\end{figure}

\chapter{System Analysis and Design}
\section{System Analysis}
\subsection{Requirement Analysis}
\begin{enumerate}
    \item \textbf{Functional Requirements}
    \begin{itemize}
        \item User authentication and authorization
        \item Schedule creation and management
        \item Automated conflict detection
        \item Resource allocation
        \item Notification system
        \item Report generation
        \item Schedule export functionality
    \end{itemize}
    
    \item \textbf{Non-Functional Requirements}
    \begin{itemize}
        \item Performance: Response time < 2 seconds
        \item Scalability: Support for 10,000 concurrent users
        \item Availability: 99.9% uptime
        \item Security: Data encryption and secure authentication
        \item Usability: Intuitive interface design
        \item Maintainability: Modular architecture
    \end{itemize}
\end{enumerate}

\subsection{Feasibility Analysis}
\begin{enumerate}
    \item \textbf{Technical Feasibility}
    \begin{itemize}
        \item Modern web technologies available
        \item Cloud infrastructure support
        \item Required development expertise
        \item Existing libraries and frameworks
    \end{itemize}
    
    \item \textbf{Operational Feasibility}
    \begin{itemize}
        \item User-friendly interface
        \item Minimal training required
        \item Automated processes
        \item Regular maintenance schedule
    \end{itemize}
    
    \item \textbf{Economic Feasibility}
    \begin{itemize}
        \item Development costs
        \item Infrastructure costs
        \item Maintenance costs
        \item Return on investment
    \end{itemize}
    
    \item \textbf{Schedule Feasibility}
    \begin{itemize}
        \item Development timeline
        \item Resource availability
        \item Milestone planning
        \item Risk management
    \end{itemize}
\end{enumerate}

\subsection{ER Diagram}
\begin{figure}[H]
\centering
\begin{tikzpicture}[node distance=2.5cm, every edge/.style={link}, scale=0.6, transform shape]
\tikzstyle{entity} = [rectangle, draw, fill=primarycolor!20, minimum width=3cm, minimum height=1.5cm]
\tikzstyle{relationship} = [diamond, draw, fill=secondarycolor!20, minimum width=2.5cm, minimum height=1.5cm]
\tikzstyle{attribute} = [ellipse, draw, fill=accentcolor!20, scale=0.7]
\tikzstyle{link} = [-latex', thick]

% Entities with more space
\node[entity] (user) at (0,0) {User};
\node[entity] (schedule) at (8,0) {Schedule};
\node[entity] (room) at (0,-6) {Room};
\node[entity] (availability) at (8,-6) {Availability};
\node[entity] (notification) at (16,0) {Notification};

% Relationships with adjusted positions
\node[relationship] (creates) at (4,0) {Creates};
\node[relationship] (uses) at (4,-3) {Uses};
\node[relationship] (sets) at (4,-6) {Sets};
\node[relationship] (receives) at (12,-3) {Receives};
\node[relationship] (triggers) at (12,0) {Triggers};

% Attributes with more spacing
\node[attribute] (userId) at (-2,1.5) {userId};
\node[attribute] (userName) at (0,1.5) {name};
\node[attribute] (userRole) at (2,1.5) {role};

\node[attribute] (scheduleId) at (6,1.5) {scheduleId};
\node[attribute] (subject) at (8,1.5) {subject};
\node[attribute] (startTime) at (10,1.5) {startTime};

\node[attribute] (roomId) at (-2,-7.5) {roomId};
\node[attribute] (roomName) at (0,-7.5) {roomName};
\node[attribute] (capacity) at (2,-7.5) {capacity};

% Draw connections with more space
\draw[link] (user) -- (creates);
\draw[link] (creates) -- (schedule);
\draw[link] (schedule) -- (uses);
\draw[link] (uses) -- (room);
\draw[link] (user) -- (sets);
\draw[link] (sets) -- (availability);
\draw[link] (user) -- (receives);
\draw[link] (receives) -- (notification);
\draw[link] (schedule) -- (triggers);
\draw[link] (triggers) -- (notification);

% Draw connections to attributes with more space
\draw (user) -- (userId);
\draw (user) -- (userName);
\draw (user) -- (userRole);

\draw (schedule) -- (scheduleId);
\draw (schedule) -- (subject);
\draw (schedule) -- (startTime);

\draw (room) -- (roomId);
\draw (room) -- (roomName);
\draw (room) -- (capacity);

% Cardinality with more space
\node[font=\small] at (2,0.2) {1};
\node[font=\small] at (6,0.2) {N};

\node[font=\small] at (6.5,-2.8) {N};
\node[font=\small] at (1.5,-2.8) {1};

\node[font=\small] at (2,-5.8) {1};
\node[font=\small] at (6,-5.8) {N};
\end{tikzpicture}
\caption{Entity Relationship Diagram}
\label{fig:erd}
\end{figure}

\subsection{Process Modeling}
\begin{figure}[H]
\centering
\begin{tikzpicture}[>=stealth]
% Define participants
\node (user) at (0,0) {User};
\node (ui) at (3,0) {UI};
\node (api) at (6,0) {API};
\node (db) at (9,0) {Database};

% Draw lifelines
\draw[dashed] (user) -- (0,-6);
\draw[dashed] (ui) -- (3,-6);
\draw[dashed] (api) -- (6,-6);
\draw[dashed] (db) -- (9,-6);

% Draw interactions
\draw[->] (0,-1) -- (3,-1) node[midway,above] {Create Schedule};
\draw[->] (3,-1.5) -- (6,-1.5) node[midway,above] {API Request};
\draw[->] (6,-2) -- (9,-2) node[midway,above] {Validate};
\draw[<-] (6,-2.5) -- (9,-2.5) node[midway,above] {Response};
\draw[->] (6,-3) -- (9,-3) node[midway,above] {Save};
\draw[<-] (3,-3.5) -- (6,-3.5) node[midway,above] {Success};
\draw[<-] (0,-4) -- (3,-4) node[midway,above] {Confirmation};

% Add activation boxes
\fill[gray!20] (2.8,-1) rectangle (3.2,-4);
\fill[gray!20] (5.8,-1.5) rectangle (6.2,-3.5);
\fill[gray!20] (8.8,-2) rectangle (9.2,-3);
\end{tikzpicture}
\caption{Schedule Creation Process}
\label{fig:process}
\end{figure}

\subsection{Data Flow Diagram}
\begin{figure}[H]
\centering
\begin{tikzpicture}[node distance=2cm, scale=0.7, transform shape]
% Define styles with larger minimum sizes
\tikzstyle{process} = [rectangle, minimum width=3cm, minimum height=1cm, text centered, draw=black, fill=primarycolor!20]
\tikzstyle{entity} = [rectangle, rounded corners, minimum width=3cm, minimum height=1cm, text centered, draw=black, fill=secondarycolor!20]
\tikzstyle{datastore} = [rectangle, minimum width=3cm, minimum height=1cm, text centered, draw=black, fill=accentcolor!20]
\tikzstyle{arrow} = [thick,->,>=stealth]

% Place nodes with more spacing
\node (user) [entity] {User};
\node (auth) [process, below=2.5cm of user] {Authentication};
\node (dashboard) [process, right=3cm of auth] {Dashboard};
\node (schedule) [process, below=2.5cm of dashboard] {Schedule Mgmt};
\node (room) [process, left=3cm of schedule] {Room Mgmt};
\node (notification) [process, right=3cm of schedule] {Notifications};
\node (db) [datastore, below=2.5cm of schedule] {Appwrite DB};

% Draw edges with more space
\draw [arrow] (user) -- (auth);
\draw [arrow] (auth) -- (dashboard);
\draw [arrow] (dashboard) -- (schedule);
\draw [arrow] (dashboard) -- (room);
\draw [arrow] (dashboard) -- (notification);
\draw [arrow] (schedule) -- (db);
\draw [arrow] (room) -- (db);
\draw [arrow] (notification) -- (db);
\draw [arrow] (db) -- (schedule);
\draw [arrow] (db) -- (room);
\draw [arrow] (db) -- (notification);
\draw [arrow] (schedule) -- (notification);

% Add labels with more space
\node [font=\scriptsize] at ($(user)!0.5!(auth) + (-1.5,0)$) {Credentials};
\node [font=\scriptsize] at ($(auth)!0.5!(dashboard) + (0,-0.4)$) {Token};
\node [font=\scriptsize] at ($(dashboard)!0.5!(schedule) + (0.6,0)$) {Actions};
\node [font=\scriptsize] at ($(schedule)!0.5!(db) + (1,0)$) {CRUD};
\end{tikzpicture}
\caption{Data Flow Diagram}
\label{fig:dataflow}
\end{figure}

\section{System Design}
\subsection{Architectural Design}
\begin{figure}[H]
\centering
\begin{tikzpicture}[node distance=2.2cm, scale=0.75, transform shape]
\tikzstyle{layer} = [rectangle, draw, fill=primarycolor!20, 
    minimum width=12cm, minimum height=1.5cm, text centered, font=\bfseries]
\tikzstyle{component} = [rectangle, draw, fill=secondarycolor!20,
    minimum width=2.5cm, minimum height=1.2cm, text centered]
\tikzstyle{arrow} = [thick,->,>=stealth]
\tikzstyle{note} = [text width=3.5cm, font=\small\itshape]

% Layers with more vertical spacing
\node (client) [layer] {Client Layer};
\node (business) [layer, below=3cm of client] {Business Logic Layer};
\node (data) [layer, below=3cm of business] {Data Access Layer};
\node (storage) [layer, below=3cm of data] {Storage Layer};

% Components with more horizontal spacing
\node (ui) [component] at ($(client) + (-4,0)$) {UI Components};
\node (auth) [component] at ($(client) + (0,0)$) {Authentication};
\node (routes) [component] at ($(client) + (4,0)$) {SvelteKit Routes};

\node (schedule) [component] at ($(business) + (-4,0)$) {Schedule Service};
\node (room) [component] at ($(business) + (-1.3,0)$) {Room Service};
\node (user) [component] at ($(business) + (1.3,0)$) {User Service};
\node (notification) [component] at ($(business) + (4,0)$) {Notification Service};

\node (sdk) [component] at ($(data) + (-3,0)$) {Appwrite SDK};
\node (query) [component] at ($(data) + (0,0)$) {Query Builder};
\node (cache) [component] at ($(data) + (3,0)$) {Cache Manager};

\node (users) [component] at ($(storage) + (-4,0)$) {Users Collection};
\node (rooms) [component] at ($(storage) + (-1.3,0)$) {Rooms Collection};
\node (schedules) [component] at ($(storage) + (1.3,0)$) {Schedules Collection};
\node (notifications) [component] at ($(storage) + (4,0)$) {Notifications Collection};

% Arrows between layers
\draw [arrow] (client) -- (business);
\draw [arrow] (business) -- (data);
\draw [arrow] (data) -- (storage);

% Annotations with more space
\node [note, right=1cm of client] {SvelteKit, Tailwind CSS, FullCalendar};
\node [note, right=1cm of business] {Business Logic, Validation, Conflict Detection};
\node [note, right=1cm of data] {Appwrite SDK, Data Transformation};
\node [note, right=1cm of storage] {Appwrite Cloud Database};
\end{tikzpicture}
\caption{System Architecture}
\label{fig:architecture}
\end{figure}

\subsection{Database Design}
The system uses Appwrite Cloud as its database solution, with the following collections and schema:

\begin{figure}[H]
\centering
\begin{tikzpicture}[scale=0.65, transform shape]
\tikzstyle{table} = [rectangle, draw, fill=primarycolor!10, 
    minimum width=10cm, minimum height=1cm, text centered, font=\bfseries]
\tikzstyle{column} = [rectangle, draw, fill=white,
    minimum width=10cm, minimum height=0.8cm, text centered, font=\small]

% Tables with more vertical spacing
\node (users) [table] at (0,0) {Users Collection};
\node (user1) [column, below=0.2cm of users] {userId: string (primary key)};
\node (user2) [column, below=0.2cm of user1] {name: string};
\node (user3) [column, below=0.2cm of user2] {email: string (unique, indexed)};
\node (user4) [column, below=0.2cm of user3] {role: enum ['admin', 'teacher', 'student']};
\node (user5) [column, below=0.2cm of user4] {isActive: boolean};
\node (user6) [column, below=0.2cm of user5] {preferences: JSON};

\node (rooms) [table] at (0,-8) {Rooms Collection};
\node (room1) [column, below=0.2cm of rooms] {roomId: string (primary key)};
\node (room2) [column, below=0.2cm of room1] {roomName: string};
\node (room3) [column, below=0.2cm of room2] {capacity: number (indexed)};
\node (room4) [column, below=0.2cm of room3] {building: string (indexed)};
\node (room5) [column, below=0.2cm of room4] {floor: number};
\node (room6) [column, below=0.2cm of room5] {features: string[]};

\node (schedules) [table] at (12,0) {Schedules Collection};
\node (schedule1) [column, below=0.2cm of schedules] {scheduleId: string (primary key)};
\node (schedule2) [column, below=0.2cm of schedule1] {className: string};
\node (schedule3) [column, below=0.2cm of schedule2] {subject: string};
\node (schedule4) [column, below=0.2cm of schedule3] {teacherId: string (foreign key)};
\node (schedule5) [column, below=0.2cm of schedule4] {roomId: string (foreign key)};
\node (schedule6) [column, below=0.2cm of schedule5] {startTime: string (ISO)};
\node (schedule7) [column, below=0.2cm of schedule6] {endTime: string (ISO)};
\node (schedule8) [column, below=0.2cm of schedule7] {dayOfWeek: string};
\node (schedule9) [column, below=0.2cm of schedule8] {recurrence: string};

\node (availability) [table] at (12,-8) {Teacher Availability};
\node (avail1) [column, below=0.2cm of availability] {availabilityId: string (primary key)};
\node (avail2) [column, below=0.2cm of avail1] {teacherId: string (foreign key)};
\node (avail3) [column, below=0.2cm of avail2] {dayOfWeek: string};
\node (avail4) [column, below=0.2cm of avail3] {startTime: string (ISO)};
\node (avail5) [column, below=0.2cm of avail4] {endTime: string (ISO)};
\node (avail6) [column, below=0.2cm of avail5] {isAvailable: boolean};
\node (avail7) [column, below=0.2cm of avail6] {note: string (optional)};

\node (notifications) [table] at (0,-16) {Notifications Collection};
\node (notif1) [column, below=0.2cm of notifications] {notificationId: string (primary key)};
\node (notif2) [column, below=0.2cm of notif1] {userId: string (foreign key)};
\node (notif3) [column, below=0.2cm of notif2] {title: string};
\node (notif4) [column, below=0.2cm of notif3] {message: string};
\node (notif5) [column, below=0.2cm of notif4] {type: string};
\node (notif6) [column, below=0.2cm of notif5] {isRead: boolean};
\node (notif7) [column, below=0.2cm of notif6] {createdAt: timestamp};
\end{tikzpicture}
\caption{Database Schema}
\label{fig:db-schema}
\end{figure}

\subsubsection{Database Indexes}
To optimize query performance, the following indexes are implemented:

\begin{itemize}
    \item \textbf{Users Collection:}
    \begin{itemize}
        \item email (unique) - For fast user lookup during authentication
        \item role - For filtering users by role
    \end{itemize}
    
    \item \textbf{Rooms Collection:}
    \begin{itemize}
        \item capacity - For finding rooms with adequate capacity
        \item building - For filtering rooms by building
        \item features - For searching rooms with specific features
    \end{itemize}
    
    \item \textbf{Schedules Collection:}
    \begin{itemize}
        \item teacherId - For finding schedules by teacher
        \item roomId - For finding schedules by room
        \item dayOfWeek - For filtering schedules by day
        \item startTime, endTime - For time-based queries
        \item Composite index (roomId, dayOfWeek, startTime, endTime) - For conflict detection
    \end{itemize}
    
    \item \textbf{Availability Collection:}
    \begin{itemize}
        \item teacherId - For finding availability by teacher
        \item dayOfWeek - For filtering availability by day
        \item Composite index (teacherId, dayOfWeek) - For efficient availability lookup
    \end{itemize}
    
    \item \textbf{Notifications Collection:}
    \begin{itemize}
        \item userId - For finding notifications for a specific user
        \item isRead - For filtering unread notifications
        \item createdAt - For sorting notifications by time
    \end{itemize}
\end{itemize}

\subsubsection{Data Relationships}
The database implements the following relationships:

\begin{itemize}
    \item One-to-Many: User to Schedules (a teacher can have multiple schedules)
    \item One-to-Many: User to Availability (a teacher can have multiple availability slots)
    \item One-to-Many: User to Notifications (a user can have multiple notifications)
    \item One-to-Many: Room to Schedules (a room can be used in multiple schedules)
    \item Many-to-Many: Teachers to Rooms (through Schedules collection)
\end{itemize}

\subsubsection{Data Security}
Appwrite provides document-level security through access control rules:

\begin{itemize}
    \item Administrators have full access to all collections
    \item Teachers can:
        \begin{itemize}
            \item Read all schedules
            \item Create and update their own availability
            \item Read room information
        \end{itemize}
    \item Students can:
        \begin{itemize}
            \item Read schedules relevant to their classes
            \item Read teacher and room information
        \end{itemize}
    \item All users can manage their own notifications
\end{itemize}

\subsection{Interface Design}
The user interface follows modern design principles:

\begin{itemize}
    \item Clean and intuitive layout
    \item Responsive design for all devices
    \item Consistent color scheme and typography
    \item Accessible components
    \item Clear navigation structure
\end{itemize}

Key interface components:
\begin{itemize}
    \item Dashboard
    \item Schedule Calendar
    \item Room Management
    \item User Management
    \item Settings Panel
\end{itemize}

\begin{figure}[H]
\centering
\begin{tikzpicture}[scale=0.6]
% Define colors
\definecolor{navcolor}{RGB}{30,41,59}
\definecolor{bgcolor}{RGB}{241,245,249}
\definecolor{cardcolor}{RGB}{255,255,255}
\definecolor{primarybtn}{RGB}{79,70,229}
\definecolor{textcolor}{RGB}{15,23,42}

% Draw browser window with more space
\draw[rounded corners=5pt, fill=bgcolor] (0,0) rectangle (16,10);

% Navigation sidebar with more width
\draw[rounded corners=5pt, fill=navcolor] (0.5,0.5) rectangle (3.5,9.5);
\node[text=white, font=\bfseries] at (2,9) {TimeTablePro};

% Nav items with more spacing
\foreach \y/\text in {8/Dashboard, 7/Schedule, 6/Rooms, 5/Teachers} {
    \node[text=white, font=\small] at (2,\y) {\text};
}

% Main content area with more space
\node[align=left, font=\bfseries] at (10,9) {Schedule Management};

% Calendar header with more width
\draw[fill=cardcolor, rounded corners=3pt] (4,8) rectangle (15.5,8.8);
\node[font=\small] at (5,8.4) {Week};
\node[font=\small] at (7,8.4) {Month};
\node[font=\small, right] at (13,8.4) {March 2025};

% Calendar grid with more space
\draw[fill=cardcolor, rounded corners=3pt] (4,0.5) rectangle (15.5,7.8);

% Calendar content with more spacing
\draw[fill=primarybtn!40, rounded corners=2pt] (5,7) rectangle (7,5);
\node[text=textcolor, align=center, font=\small] at (6,6) {Database};

\draw[fill=orange!40, rounded corners=2pt] (8,5) rectangle (10,3);
\node[text=textcolor, align=center, font=\small] at (9,4) {Software};

\draw[fill=green!40, rounded corners=2pt] (11,6) rectangle (13,4);
\node[text=textcolor, align=center, font=\small] at (12,5) {Web Dev};

\draw[fill=red!40, rounded corners=2pt] (14,4) rectangle (15,2);
\node[text=textcolor, align=center, font=\small] at (14.5,3) {Networks};

% Add button with more space
\draw[fill=primarybtn, rounded corners=5pt] (14.5,8.2) rectangle (15.5,8.6);
\node[text=white] at (15,8.4) {+};
\end{tikzpicture}
\caption{Schedule Management Interface}
\label{fig:ui-mockup}
\end{figure}

\chapter{Testing}
\section{Test Cases for Unit Testing}
\begin{longtable}{|p{3cm}|p{4cm}|p{4cm}|p{4cm}|}
\hline
\textbf{Component} & \textbf{Test Case} & \textbf{Expected Result} & \textbf{Status} \\
\hline
Authentication & Valid login credentials & Successful login & Passed \\
\hline
Authentication & Invalid password & Error message & Passed \\
\hline
Schedule Creation & Valid schedule data & Schedule created & Passed \\
\hline
Conflict Detection & Overlapping schedules & Conflict detected & Passed \\
\hline
Room Assignment & Valid room selection & Room assigned & Passed \\
\hline
\end{longtable}

\section{Test Cases for System Testing}
\begin{longtable}{|p{3cm}|p{4cm}|p{4cm}|p{4cm}|}
\hline
\textbf{Feature} & \textbf{Test Scenario} & \textbf{Expected Result} & \textbf{Status} \\
\hline
End-to-End Schedule & Create and view schedule & Schedule visible to all users & Passed \\
\hline
Notification System & Schedule update & Notifications sent & Passed \\
\hline
Data Export & Export to PDF & Valid PDF generated & Passed \\
\hline
Load Testing & 1000 concurrent users & Response time < 2s & Passed \\
\hline
Security & SQL injection attempt & Attack prevented & Passed \\
\hline
\end{longtable}

\section{Performance Testing Results}
\begin{figure}[H]
\centering
\begin{tikzpicture}[scale=0.8]
\begin{axis}[
    xlabel={Number of Users},
    ylabel={Response Time (ms)},
    xmin=0, xmax=100,
    ymin=0, ymax=1000,
    xtick={0,25,50,75,100},
    ytick={0,250,500,750,1000},
    legend pos=north west,
    ymajorgrids=true,
    grid style=dashed,
    width=12cm,
    height=8cm,
    legend style={font=\small},
    tick label style={font=\small},
    label style={font=\small}
]
\addplot[
    color=primarycolor,
    mark=square,
    ]
    coordinates {
    (0,100)(25,200)(50,350)(75,550)(100,850)
    };
\addplot[
    color=secondarycolor,
    mark=triangle,
    ]
    coordinates {
    (0,80)(25,120)(50,180)(75,280)(100,450)
    };
\legend{Without Optimization,With Optimization}
\end{axis}
\end{tikzpicture}
\caption{Performance Test Results}
\label{fig:performance}
\end{figure}

\chapter{Conclusion and Future Recommendations}
\section{Lesson Learned / Outcome}
Key learnings from the project:
\begin{itemize}
    \item Importance of proper requirement analysis
    \item Value of automated testing
    \item Need for scalable architecture
    \item Significance of user feedback
    \item Impact of proper documentation
\end{itemize}

\section{Conclusion}
TimeTablePro successfully addresses the challenges of modern timetable management \cite{user_experience}:
\begin{itemize}
    \item Automated scheduling reduces manual effort
    \item Real-time updates improve communication
    \item Conflict detection prevents scheduling errors \cite{conflict_detection}
    \item Intuitive interface enhances user experience
    \item Scalable architecture ensures future growth
\end{itemize}

\section{Future Recommendations}
Proposed future enhancements:
\begin{itemize}
    \item Mobile application development
    \item Advanced AI-based scheduling
    \item Integration with other systems
    \item Enhanced reporting features
    \item Offline functionality
\end{itemize}

\begin{figure}[H]
\centering
\begin{tikzpicture}[scale=0.8]
\begin{axis}[
    ybar,
    xlabel={Development Priorities},
    ylabel={Priority Score (1-10)},
    symbolic x coords={Mobile,AI,Integration,Reports,Offline},
    xtick=data,
    nodes near coords,
    nodes near coords align={vertical},
    width=12cm,
    height=8cm,
    bar width=20pt,
    ymin=0,
    ymax=10,
    axis lines*=left,
    ymajorgrids=true,
    grid style=dashed,
    legend style={font=\small},
    tick label style={font=\small},
    label style={font=\small}
]
\addplot[fill=primarycolor!40] coordinates {
    (Mobile,8)
    (AI,7)
    (Integration,6)
    (Reports,5)
    (Offline,4)
};
\end{axis}
\end{tikzpicture}
\caption{Future Development Priorities}
\label{fig:priorities}
\end{figure}

\end{document} 